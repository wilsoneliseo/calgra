\section{Archivo io.mac}
\label{sec-1}
Macros para entrada y salida utilizando interrupciones. Son
ejecutables en DOS. Para emular DOS se utilizó DOSBox. Es ensamblado
por NASM.
  
\subsection{TERMINAR\_PROGRAMA}
\label{sec-1-1}
Salir del programa
\begin{itemize}
\item Entrada
\begin{description}
\item[vacio] nada
\end{description}
\item Salida
\begin{description}
\item[vacio] nada
\end{description}
\end{itemize}

\subsection{PRINT}
\label{sec-1-2}
Imprimir cadena en salida estandar.
\begin{itemize}
\item Entrada
\begin{description}
\item[dx,\%1] puntero a cadena que termina en ''\$''
\end{description}
\item Salida
\begin{description}
\item[vacio] nada
\end{description}
\end{itemize}

\subsection{PRINT\_CHAR}
\label{sec-1-3}
Imprimir caracter en la salida estandar
\begin{itemize}
\item Entrada
\begin{description}
\item[dl,\%1] el caracter a imprimir
\end{description}
\item Salida
\begin{description}
\item[al] la ultima salida de caracter
\end{description}
\end{itemize}
\href{http://www.ctyme.com/intr/rb-2554.htm}{ir}

\subsection{ABRIR\_ARCHIVO}
\label{sec-1-4}
Abre un archivo existente.
\begin{itemize}
\item Entrada
\begin{description}
\item[dx,\%1] puntero a cadena con nombre/ruta del archivo a
abrir. Este nombre debe finalizar en 0
\end{description}
\item Salidas
\begin{description}
\item[CF=0] archivo se abrio exitosamente. En AX el handle
del archivo
\item[CF=1] ocurrio algun error. El código del error en AX
\end{description}
\end{itemize}
\href{http://www.ctyme.com/intr/rb-2779.htm}{ir}

\subsection{LEER\_ARCHIVO}
\label{sec-1-5}
Lee archivo abierto.
\begin{itemize}
\item Entradas
\begin{description}
\item[bx,\%1] handle del archivo
\item[cx,\%2] numero de bytes a leer
\item[dx,\%3] buffer en donde guardar lo leido
\end{description}
\item Salidas
\begin{description}
\item[CF=0] se leyo exitosamente. En AX el numero de bytes
actualmente leídos
\item[CF=1] ocurrio algun error. El código del error en AX
\end{description}
\end{itemize}
\href{http://www.ctyme.com/intr/rb-2783.htm}{ir}


\subsection{ESENAR}
\label{sec-1-6}
EScribir EN un ARchivo abierto
\begin{itemize}
\item Entradas
\begin{description}
\item[bx,\%1] handle del archivo
\item[cx,\%2] numero de bytes a escribir
\item[dx,\%3] puntero a array que se escribirá
\end{description}
\item Salidas
\begin{description}
\item[CF=0] se escribió exitosamente. En AX el numero de
bytes actualmente escritos.
\item[CF=1] ocurrio algun error. El código del error en AX
(05h,06h).
\end{description}
\end{itemize}
\href{http://www.ctyme.com/intr/rb-2791.htm}{ir}


\subsection{CERRAR\_ARCHIVO}
\label{sec-1-7}
Cierra un archivo
\begin{itemize}
\item Entrada
\begin{description}
\item[bx,\%1] handle del archivo a cerrar
\end{description}
\item Salida
\begin{description}
\item[CF=0] se cerro exitosamente. AX destruido
\item[CF=1] ocurrio algun error al cerrar. En AX el código
de error (06h).
\end{description}
\end{itemize}

\subsection{CREAR\_ARCHIVO}
\label{sec-1-8}
Crear un archivo nuevo o truncar si ya existe.
\begin{itemize}
\item Entrada
\begin{description}
\item[dx,\%1] puntero a cadena que será nombre del
archivo. Este nombre debe terminar en 0.
\end{description}
\item Salida
\begin{description}
\item[CF=0] se creo exitosamente. En AX el handle de
archivo.
\item[CF=1] ocurrio algun error al crear En AX el código
de error (03h,04h,05h).
\end{description}
\end{itemize}
\href{http://www.ctyme.com/intr/rb-2778.htm#Table1401}{ir}


\subsection{UBICAR\_CURSOR}
\label{sec-1-9}
ubicar cursor en la pantalla cuando se esta en modo
texto.
\begin{itemize}
\item Entradas
\begin{description}
\item[dh,\%1] fila
\item[dl,\%2] columna
\end{description}
\end{itemize}


\subsection{LIMPIAR\_PANTALLA}
\label{sec-1-10}
Limpiar pantalla cuando se esta en modo texto.
\begin{itemize}
\item Entrada
\begin{description}
\item[vacio] nada
\end{description}
\item Salida
\begin{description}
\item[vacio] nada
\end{description}
\end{itemize}


\subsection{DESPLAZAR\_ARRIBA}
\label{sec-1-11}
Desplaza hacia arriba. Es como borrar la pantalla según los
parámentros explicados a  continuación:
\begin{itemize}
\item Entrada
\begin{description}
\item[ch,\%1] fila superior izquierda
\item[cl,\%2] columna superior izquierda
\item[dh,\%3] fila inferior derecha
\item[dl,\%4] columna inferior derecha
\item[al,\%5] número de líneas a desplazarse hacia arriba
\end{description}
\item Salida
\begin{description}
\item[vacio] nada
\end{description}
\end{itemize}

\href{http://www.ctyme.com/intr/rb-0096.htm}{ir}


\subsection{GETCHAR}
\label{sec-1-12}
Obtiene ascii de tecla presionada sin eco, via BIOS.
\begin{itemize}
\item Entrada
\begin{description}
\item[vacio] nada
\end{description}
\item Salida
\begin{description}
\item[AL] en este registro se almacena el caracter leido
\end{description}
\end{itemize}

\subsection{READ\_FOR\_BUFF\_INPUT}
\label{sec-1-13}
lee cadena de entrada estandar y lo guarda un arreglo. Se
refirá a este arreglo como buffInput, de aqui en adelante. 
\begin{itemize}
\item Entradas
\begin{description}
\item[dh,\%1] fila
\item[bx,\%2] columna
\item[dx,\%3] arreglo donde guardar la entrada del usuario
\end{description}
\item Salida
\begin{itemize}
\item arreglo buffInput con la entrada del usuario. La
interrupcion utilizada deja dicho arreglo en el
siguiente formato, suponiedo que el usuario ingreso
la cadena ''cad'' en la entrada estandar:
\begin{verbatim}
+--+---+---+---+---+----+---+
|41| 3 | c | a | d | \r |.. |
+--+---+---+---+---+----+---+
\end{verbatim}

donde:
\begin{itemize}
\item\relax [buffInput] contiene 41, indica el numero maximo de
chars que puede guardar
\item\relax [buffInput+1] contiene 3 indica el numero de chars
ingresados por el usuario
\item\relax [buffInput+2] contiene ''c'' es el primer char de la
cadena ingresada. La siguientes dos posiciones tiene
''a'' y ''d''
\item\relax [buffInput+5] contiene el retorno de carro. El
contenido del resto del arreglo esta indefinido
\end{itemize}
\end{itemize}
\end{itemize}

\subsection{READ\_FOR\_BUFF\_INPUT}
\label{sec-1-14}
Lee cadena de entrada estandar y lo guarda un arreglo. NO
necesita de fila y columna.
\begin{itemize}
\item Entradas
\begin{description}
\item[dx,\%1] arreglo donde guardar la entrada del usuario
\end{description}
\item Salida
\begin{itemize}
\item arreglo buffInput con la entrada del usuario.
\end{itemize}
\end{itemize}

\subsection{PRINT}
\label{sec-1-15}
Imprimir cadena en salida estandar en una fila y columna
especificada.
\begin{itemize}
\item Entrada
\begin{description}
\item[dh,\%1] fila
\item[dl,\%2] columna
\item[dx,\%3] puntero a cadena que termina en ''\$''
\end{description}
\item Salida
\begin{description}
\item[vacio] nada
\end{description}
\end{itemize}

\subsection{GETCHAR}
\label{sec-1-16}
Obtiene ascii de tecla presionada, aunque se emplea para
parar el programa, hasta que el usuario presione alguna
tecla, mostrando un mensaje que venga al caso.
\begin{itemize}
\item Entrada
\begin{description}
\item[\%1] Puntero a cadena que se mostrara.
\end{description}
\item Salida
\begin{description}
\item[AL] en este registro se almacena el caracter leido
\end{description}
\end{itemize}

\section{Archivo anlex2.asm}
\label{sec-2}
Contiene la funcion ''scanner'' que realiza el analisis lexico de una
funcion polinomica de grado 4 con coeficientes de dos digitos
positivos o negativos. Un termino correcto debe iniciar con ''+'' o ''-''
seguido de cero, uno o dos digitos. Luego puede venir o no una ''x'' o
bien ''x'' seguido por ''2'' o ''3'' o ''4''. La expresion regular es la
siguiente:
\begin{verbatim}
ER=(('+'|'-')(<d>?<d>)?[x^<expo>|x]?)+;
donde: <d>=0|1|2|3|4|5|6|7|8|9 y <expo>=2|3|4
\end{verbatim}

Estos son algunos ejemplos de cadenas lexicamente correctas para este
lenguaje (debe terminar en '';''):
\begin{verbatim}
+2x+1-99x  +29  +x +29;
-90x^3 +5  5x^2;
\end{verbatim}


\subsection{scanner}
\label{sec-2-1}

Realiza un analisis lexico de la cadena terminada en '';''. Muestra fila
y columna si ocurre error. Muestra exito si se reconoce la
cadena. Guarda los exponentes y coeficientes en una arreglo de
estructuras.

\begin{itemize}
\item Parametro
\begin{description}
\item[[BP+6]] puntero al primer elemento del arreglo de chars a
analizar
\item[[BP+4]] puntero a estructura que guarda coeficientes y
exponentes y otros datos de la funcion.
\end{description}
\item Variables Locales
\begin{enumerate}
\item\relax [BP-2]  fila
\item\relax [BP-4]  columna
\item\relax [BP-6]  entrada
\item\relax [BP-8]  estado
\item\relax [BP-0xA]  exponente
\item\relax [BP-0xC]  primer char ascii de un coeficiente
\item\relax [BP-0xE]  segundo char ascii de un coeficiente
\item\relax [BP-0x10]  tercer char ascii de un coeficiente
\item\relax [BP-0x12]  numero de chars que forman el coeficiente
\item\relax [BP-0x14] contador de terminos por funcion
\end{enumerate}
\end{itemize}


\subsection{scanner\_f}
\label{sec-2-2}
Lo mismo que scanner solo que trabaja con un archivo.

\begin{itemize}
\item Parametro
\begin{description}
\item[[BP+6]] handle del archivo de donde lee
\item[[BP+4]] puntero a estructura que guarda coeficientes y
exponentes y otros datos de la funcion.
\end{description}
\item Variables Locales
\begin{enumerate}
\item\relax [BP-2]  fila
\item\relax [BP-4]  columna
\item\relax [BP-6]  entrada
\item\relax [BP-8]  estado
\item\relax [BP-0xA]  exponente
\item\relax [BP-0xC]  primer char ascii de un coeficiente
\item\relax [BP-0xE]  segundo char ascii de un coeficiente
\item\relax [BP-0x10]  tercer char ascii de un coeficiente
\item\relax [BP-0x12]  numero de chars que forman el coeficiente
\item\relax [BP-0x14] contador de terminos por función
\end{enumerate}
\end{itemize}

\subsection{rutinaDeError}
\label{sec-2-3}
Imprime un mensaje de error lexico.
\begin{itemize}
\item Parametros
\begin{description}
\item[[bp+8]] Fila en binario
\item[[bp+6]] Columna en binario
\item[[bp+4]] El caracter ascii que causo el error
\end{description}
\item Entrada
\begin{description}
\item[bx] puntero a cadena que describe el error
\end{description}
\item Salidas
\begin{description}
\item[vacio] nada
\end{description}
\end{itemize}

\subsection{convNumStrParaNumBin}
\label{sec-2-4}
Convertir numero en string a numero en binario para el manejo interno
como cantidad de la representacion numerica de la cadena. Se utilizó
para leer la cadena directamente de la pila.

\begin{itemize}
\item Entradas
\begin{description}
\item[di] puntero al primer elemento del arreglo de caracteres
(cadena) a convertir. Cada caracter debe estar a cada dos
bytes. Esto es porque se utilizó para leer de la pila cuyos
elementos son de 2 bytes.

\item[cl] tamaño de la cadena apuntada por bx
\end{description}

\item Salida
\begin{description}
\item[ax] el numero de 1 word en binario. En complemento a dos si es
negativo.
\end{description}

\item Variable local
\begin{description}
\item[[BP-2]] variable temporal (temp) que guarda la cantidad que
finalmente se guarda en ax al terminar el proceso
\end{description}
\end{itemize}

\subsection{convNumStrParaNumBin\_arr}
\label{sec-2-5}
Igual que ''convNumStrParaNumBin'' solo que el array de caracteres deben
estar a cada byte no a cada dos, esto para que no necesariamente deba
leerse desde la pila.
\begin{itemize}
\item Entradas
\begin{description}
\item[di] puntero al primer elemento del arreglo de caracteres
(cadena) a convertir.
\item[cl] tamaño de la cadena apuntada por di
\end{description}

\item Salida
\begin{description}
\item[ax] el numero de 1 word en binario. En complemento a dos si es
negativo.
\end{description}

\item Variable local
\begin{description}
\item[[BP-2]] variable temporal (temp) que guarda la cantidad que
finalmente se guarda en ax al terminar el proceso
\end{description}
\end{itemize}

\subsection{printNumBin}
\label{sec-2-6}
Imprimir numero binario en la salida estandar
\begin{itemize}
\item Entradas
\begin{description}
\item[ax] el numero binario (word) a imprimir
\end{description}
\item Salidas
\begin{description}
\item[vacio] nada
\end{description}
\end{itemize}


\subsection{printNumBin\_mas}
\label{sec-2-7}
Lo mismo que printNumBin solo que los positivos les antepone el signo
positivo ''+''
\begin{itemize}
\item Entradas
\begin{description}
\item[ax] el numero binario (word) a imprimir
\end{description}
\item Salidas
\begin{description}
\item[vacio] nada
\end{description}
\end{itemize}


\section{Archivo ploter.asm}
\label{sec-3}

\subsection{dibujar\_plano}
\label{sec-3-1}
dibuja los ejes X y Y del plano cartesiano
\begin{itemize}
\item Parámetros
\begin{description}
\item[[BP+6]] posición x del origen
\item[[BP+4]] posición y del origen
\end{description}
\end{itemize}

\subsection{putPixel}
\label{sec-3-2}
Pinta un pixel en cordenada y color dado. Las coordenadas
son con respecto al origen del plano establecido durante la
llamada ha ''dibujar\_plano''.
\begin{itemize}
\item Entradas
\begin{description}
\item[dl] color
\item[bx] coordenada x
\item[ax] coordenada y
\end{description}
\end{itemize}

\subsection{dibujar\_linea\_horizontal}
\label{sec-3-3}
dibuja una linea horizontal
\begin{itemize}
\item Parámetros
\begin{description}
\item[[BP+10]] color
\item[[BP+8]] ancho
\item[[BP+6]] x
\item[[BP+4]] y
\end{description}
\end{itemize}


\subsection{dibujar\_linea\_vertical}
\label{sec-3-4}
Dibuja una linea vertical
\begin{itemize}
\item Parámetros
\begin{description}
\item[[BP+10]] color
\item[[BP+8]] alto
\item[[BP+6]] x
\item[[BP+4]] y
\end{description}
\end{itemize}

\subsection{dibujar\_rectangulo}
\label{sec-3-5}
Dibuja un rectangulo relleno de un color dado.
\begin{itemize}
\item Parámetros
\begin{description}
\item[[BP+12]] color
\item[[BP+10]] ancho
\item[[BP+8]] alto
\item[[BP+6]] x
\item[[BP+4]] y
\end{description}
\end{itemize}

\section{Archivo rep.asm}
\label{sec-4}
\subsection{printPol\_f}
\label{sec-4-1}
Imprimir polinomio en archivo
\begin{itemize}
\item Parametros
\begin{description}
\item[[BP+6]] handle del archivo
\item[[BP+4]] puntero al inicio del arreglo que contiene la
expresion polinomica
\end{description}
\end{itemize}

\subsection{printCoef\_f}
\label{sec-4-2}
Imprimir coeficiente en archivo
\begin{itemize}
\item Parametros
\begin{description}
\item[[PB+8]] handle del archivo
\item[[BP+6]] coeficiente en binario (word)
\item[[BP+4]] exponente en binario (byte)
\end{description}
\end{itemize}

\subsection{printNumBin\_f}
\label{sec-4-3}
Imprimir numero binario en archivo
\begin{itemize}
\item Parametro
\begin{description}
\item[[BP+6]] valor para saber si se quiere imprimir los
positivos anteponiendo un ''+''. Si tiene el
valor de POS\_CON\_MAS se imprime el ''+''
\item[[BP+4]] handle del archivo
\end{description}
\item Entradas
\begin{description}
\item[ax] el numero binario (word) a imprimir
\end{description}
\item Variables locales
\begin{description}
\item[[BP-2]] un temporal para ax
\item[[BP-4]] un temporal para bx
\item[[BP-6]] un temporal para cx
\end{description}
\end{itemize}

\section{Archivo calgra.asm}
\label{sec-5}

\subsection{evalPol}
\label{sec-5-1}
Evaluar polinomio de 5 terminos como maximo
\begin{itemize}
\item Entradas:
\begin{description}
\item[di] puntero a estructura que contiene el
polinomio a evaluar
\end{description}
\item Parámetros
\begin{description}
\item[[PB+4]] número a evaluar de un word (x)
\end{description}
\item Variables locales
\begin{description}
\item[[BP-2]] temporal que al final será ax
\item[[BP-4]] contador para bucle mayor
\end{description}
\item Salidas
\begin{description}
\item[ax] el resultado de la evaluación (y)
\end{description}
\end{itemize}


\subsection{printInfoEsquina}
\label{sec-5-2}
imprimir cantidad de polinomios actuales en el sistema y
cuanto es el maximo que puede ingresarse. Esto en la
esquina inferior derecha.


Crear reportes.
\begin{itemize}
\item Variables locales
\begin{description}
\item[[BP-2]] handle de archivo ya sea el creado para
polinomios del sistema o bien el creado para
ecuaciones resueltas. Es uno o el otro nunca
ámbos.
\item[[BP-4]] contador del bucle
\item[[BP-6]] para llevar la letra de cada funcion
\item[[BP-8]] temporal para la evaluacion de puntos
\item[[BP-10]] igual que el anterior
\end{description}
\end{itemize}

\subsection{setOrientacionPol}
\label{sec-5-3}
Establecer si un polinomio de grado 1,2,3 o 4, es positivo
o negativo. Para uso exclusivo de rutina \texttt{conocerGrado}.
\begin{itemize}
\item Entradas
\begin{description}
\item[bx] coeficiente del termino de exponenete mayor
\end{description}
\item Salidas
\begin{description}
\item[ah] POL\_POSITIVO si positivo y POL\_POSITIVO-1 si
negativo
\end{description}
\end{itemize}


\subsection{conocerGrado}
\label{sec-5-4}
Saber el grado de un polinomio
\begin{itemize}
\item Entradas
\begin{description}
\item[di] puntero a estructura que contiene el polinomio
\end{description}
\item Salidas
\begin{description}
\item[al] grado del polinomio
\item[ah] POL\_POSITIVO si positivo y POL\_POSITIVO-1 si
negativo
\end{description}
\end{itemize}


\subsection{Graficar polinomios}
\label{sec-5-5}
\begin{itemize}
\item Variables locales
\begin{description}
\item[[bp-2]] contador temporal para bucle mayor
\item[[bp-4]] limite inferior del rango a graficar (x0)
\item[[bp-6]] limite superior del rango a graficar (x1)
\item[[bp-8]] factor por el cual se dividen los puntos en
el eje Y, dependiendo si se trata de un
polinomio de 1,2,3 o 4 grado, esto para que
se logre la visualización más adecuada de la
gráfica considerando que se tiene solo 200
pixeles del modo de video para el eje Y.
\end{description}
\end{itemize}


\subsection{printIntegral}
\label{sec-5-6}
Imprime una integral en salida estandar. Basicamente es lo
mismo que printPol, la diferencia es que antes de hacer la
llamada a printPol imprime la notacion de integrales.
\begin{itemize}
\item Entrada
\begin{description}
\item[si] polinomio integrado
\end{description}
\end{itemize}


\subsection{printDerivada}
\label{sec-5-7}
Imprime una derivada en salida estandar. Basicamente es lo
mismo que printPol, la diferencia es que antes de hacer la
llamada a printPol imprime la notacion de derivadas.
\begin{itemize}
\item Entrada
\begin{description}
\item[si] polinomio derivado de otro polinomio
\end{description}
\end{itemize}


\subsection{printPolinomio}
\label{sec-5-8}
Imprime una integral en salida estandar. Basicamente es lo
mismo que printPol, la diferencia es que antes de hacer la
llamada a printPol imprime la notacion de funcion
identificandola con una letra mayuscula.
\begin{itemize}
\item Entrada
\begin{description}
\item[di] polinomio original que no es derivada ni integral
de otro polinomio
\end{description}
\end{itemize}


\subsection{printCoef}
\label{sec-5-9}
Imprimir coeficiente en la salida estandar
\begin{itemize}
\item Parametros
\begin{description}
\item[[BP+6]] coeficiente en binario (word)
\item[[BP+4]] exponente en binario (byte)
\end{description}
\end{itemize}

\subsection{printPol}
\label{sec-5-10}
Imprimir polinomio en salida estandar
\begin{itemize}
\item Parametros
\begin{description}
\item[[BP+4]] puntero al inicio del arreglo que contiene la
expresion polinomica
\end{description}
\end{itemize}

\subsection{integrarPol}
\label{sec-5-11}
Integra una polinomio de 5 terminos como maximo
\begin{itemize}
\item Entrada:
\begin{description}
\item[di] puntero a estructura que contiene el polinomio a
integrar.
\item[si] puntero a estructura donde guardar la integración.
\end{description}
\end{itemize}

\subsection{integrarPol}
\label{sec-5-12}
Integra una polinomio de 5 terminos como maximo
\begin{itemize}
\item Entrada:
\begin{description}
\item[di] puntero a estructura que contiene el polinomio a
derivar.
\item[si] puntero a estructura donde guardar la derivación.
\end{description}
\end{itemize}


\subsection{ing}
\label{sec-5-13}
Opcion ''ingresar'' del menú principal, cuyo objetivo es
cargar un archivo con expresiones polinómicas terminadas en
'';''
\begin{itemize}
\item Variables Locales
\begin{description}
\item[[BP-2]] guarda el handle del archivo abierto.
\end{description}
\end{itemize}


%%% Local Variables:
%%% mode: latex
%%% TeX-master: "tec"
%%% End:
